\section*{Concluzie}
\phantomsection
Pentru realizarea sarcinii propuse in lucrarea de laborator numarul 3, am ales softul Visual Studio 2015 si anume Windows Forms in limbajul 
\textsc{\large C\#}. Dupa pararea mea este unul dintre cele mai usoare si mai simple limbaje de programare pentru asa sarcine. Pentru inceput m-am familiarizat cu functiile de baza ale softului si cu limbajul \textsc{\large C\#}. Am facut cunostinta cu sintaxa \textsc{\large C\#} si cu meniurile Windows Forms. El are un bogat nivel de functii care se poate de utilizat pentru a crea softuri mai dificile. Am inceput crearea propriului meu calculator. De la inceput am amplasat toate cele necesare pe fereastra principala(button, label, editbox e.t.c) si am editat designul dupa gustul propriu. Calculatorul este facut dupa FLAT design cum este calculatorul din Windows 10. Dupa ce am amplasat toate cele nesare am inceput scrierea functionalului. Pentru aceasta m-am aprofundat mai adinc in functionalul si sintaxa \textsc{\large C\#}. Am cautat tutoriale pe Youtube si Google pentru a putea ajunge la realizarea finala a sarcinii dorite. Dupa un timp scurt, functionalul de baza ca + - * / a fost implementat cu succes. Mai departe am fost nevoit sa implementez si operatiile ca ridicare la putere, radical , plus/minus si operatiile cu numere zecimale. Dupa inca un timp anumit, functionalul a fost cu succes implementat. Etapa urmatoare a fost gasirea bugurilor, erorilor si eliminarea lor. Dupa ce am facut totul, calculatorul meu cu succes poate functiona fara nici o greseala.
In general aceasta lucrare de laborator mi-a permis sa ma acomodez cu lucrul in Windows Forms si cel mai principal, sa fac cunostinta cu limbajul \textsc{\large C\#}. Pot spune ca este un limbaj interesant in comparatie cu cunoscutul limbaj de mine C++ sau C. Chiar daca a fost conceput ca un concurent pentru limbajul Java, el poate fi folosit in multe sisteme, pentru ca este integrat in platforma .NET Framework. El simplifica mult scrierea de programe pentru sistemul de operare Windows. Pot spune ca poate il voi folosi in viitor ca inlocuire pentru C++. Despre laboratorul curent pot spune ca a fost foarte util pentru intelegerea lucrului cu forme (butoane, editbox, label) si scrierea functionalului pentru ele.
\clearpage